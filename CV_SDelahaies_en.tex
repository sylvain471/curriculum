% Preamble
% Compile with XeLateX
% xelatex -interaction=nonstopmode CV_SDelahaies_fr.tex 

\documentclass[11 pt,oneside,a4paper,titlepage]{article}
\usepackage{preamble}
\usepackage{enumitem}
\usepackage{amsmath,amssymb}


%\usepackage{caladea}
%\usepackage[T1]{fontenc}
\usepackage[latin9]{inputenc}
\usepackage{setspace}

%\usepackage[T1]{fontenc}
%\usepackage[onehalfspacing]{setspace}

\graphicspath{{PIC/}}
%%%%%%%%%%%%%%%%%%%%%%%%%%%%%%%%%%%%%%%%%%%%%%%%%%%%%%%%%%%%%%%%%%%%%%%%%%%%%%%%%%%%%%
\begin{document}

\sidebar{sideBarColor!25}
\simpleheader{titleBackColor}{Sylvain}{Delahaies}{Data Scientist | Data Engineer | ML Engineer}{white}{QR_En.png}{https://sdelahaies.fr/index_en.html}{photo1.jpg}

% Start Minipages
\vspace*{2.99cm}% start 8 cm from the top of the page}
    \adjustbox{valign=t}{\begin{minipage}{6.3cm} % large 7.4 cm from the top
    \vspace*{1.2cm} % text starts 1cm under the top of the minipage

        %%%%%%%%%%%%%%%%%%%%%%%%%%%%%%%%%%%%%%%%%%%%%%%%%%%%
        % Profile section
        \vspace*{0.22cm}
        \ruleline{\textbf{In brief}}
        %Mathématicien, après plus de dix ans de recherche fondamentale et appliquée dans la data j'ai repris un parcours de formation en Data Science et Data Engineering. Rompu à la veille scientifique et technologique, à la conception et à l’implémentation de méthodes innovantes pour exploiter les données, mon objectif est de concevoir, développer et déployer des solutions d'IA.\\
        %Mathématicien, après plus de dix ans de recherche fondamentale et appliquée dans la data j'ai repris un parcours de formation en Data Engineering. Rompu à la conception et à l’implémentation de méthodes innovantes pour exploiter les données, mon objectif est de concevoir, développer et déployer des solutions d'IA.\\
        Mathematician, after over ten years of academic research in fundamental and applied data sciences, I have embarked on a path in Data Engineering and AI. Experienced in designing and implementing innovative methods for data exploitation, I design, develop, and deploy AI solutions.

        \vspace*{0.22cm}
                
            
        %%%%%%%%%%%%%%%%%%%%%%%%%%%%%%%%%%%%%%%%%%%%%%%%%%%
        % Contact Section
        \ruleline{\textbf{Personal Info}}
        \vspace*{0.02cm}

        \begin{tikzpicture}[every node/.style={inner sep=0pt, outer sep=0pt}]
        \matrix [
        column 1/.style={anchor=center,contactIcon},
        column 2/.style={anchor=west,align=left,contactIcon},
        column sep=5pt,
        row sep=5pt] (contact) {
         \node{\faEnvelopeO}; & \node{\href{mailto:sdelahaies@gmail.com}{sdelahaies@gmail.com}};\\
         \node{\includegraphics[height=0.7cm]{www.png}}; & \node{Web site: \href{https://sdelahaies.fr}{sdelahaies.fr}};\\
         \node{\includegraphics[height=0.5cm]{logo-github.png}}; & \node{Github: \href{https://github.com/sdelahaies}{sdelahaies}};\\
         \node{\includegraphics[height=0.5cm]{linkedin.png}}; & \node{Linkedin: \href{https://www.linkedin.com/in/sylvaindelahaies}{sylvaindelahaies}};\\
         \node{\includegraphics[height=0.7cm]{researchgate-svgrepo-com.png}}; & \node{\href{https://www.researchgate.net/profile/Sylvain-Delahaies-2/}{Research Gate}};\\
        \node{\faMapMarker}; & \node{\href{https://wikimapia.org/#lang=fr&lat=45.118872&lon=1.236005&z=15&m=o}{Condat-sur-Vézère}, Dordogne, FR};\\};
        \end{tikzpicture} 
        \vspace*{-0.22cm}
        %%%%%%%%%%%%%%%%%%%%%%%%%%%%%%%%%%%%%%%%%%%%%%%%%%%
        \ruleline{\textbf{Languages Skills}}
        \vspace*{0.02cm}

        \begin{tikzpicture}[every node/.style={inner sep=0pt, outer sep=0pt}]
        \matrix [
        column 1/.style={anchor=center,contactIcon},
        column 2/.style={anchor=west,align=left,contactIcon},
        column sep=5pt,
        row sep=5pt] (contact) {
        \node{\flag{France.png}};
        & \node{French - Native};\\
        \node{\flag{England.png}};
        & \node{English - Professional Proficiency};\\
        };
        \end{tikzpicture} 
        \vspace*{0.22cm}

        \ruleline{\textbf{Education}}
        \begin{itemize}[leftmargin=*]
         \item {\textbf{Data Engineering} \textit{Jan-Apr 2023}\\
         DataScientest, Paris.}   
         \item {\textbf{PhD} \textit{2005\--2008} \\
        Mathematics \& Fluid Dynamics\\
        University of Surrey}
        \item {\textbf{MSc} \textit{2004\--2005} \\
        Hydro-Informatics\\
        University of Surrey}
        \item{\textbf{MPhil} \textit{1999\--2004} \\
        Mathématiques et Applications\\
        Université d'Angers}
        \end{itemize}

        %%%%%%%%%%%%%%%%%%%%%%%%%%%%%%%%%%%%%%%%%%%%%%%%%%%
        % Other Interests
        \ruleline{\textbf{Interests \& Activities}}
        % \begin{multicols}{2}
        % \begin{itemize} [leftmargin=*]
        %     \item  Municipal Elected Official 
        %     \item  Heritage Association Vice-President  
        %     \item  Technologies
        %     \item  Piano, Guitare
        %     \item  Lutherie
        %     \item  Informatique
        %     \item  Linux
        %     \item  Rénovation
        %     \item  Randonnée
        %     \item  Patrimoine
        % \end{itemize}
        % \end{multicols}
        \begin{itemize}[leftmargin=*]
            \item{Municipal Elected Official}
            \item{Vice-President of the Association \textit{Commanderie de Condat} Cultural Heritage \href{https://commanderiecondat.fr/}{CCCH}}
            \item{Technologies, computing, Linux, Artificial Intelligence}
            \item{Piano, Guitar, Lutherie}
        \end{itemize}
    \end{minipage}} %
    \hfill 
%%%%%%%%%%%%%%%%%%%%%%%%%%%%%%%%%%%%%%%%%%%%%%%%%%%%%%%%%
%%%%% MAIN SECTION %%%%%%%%%%%%%%%%%%%%
    \adjustbox{valign=t}{\begin{minipage}{12.3cm}
        \vspace*{1cm}
            % Information Technology Skills
        \section*{SKILLS}
        \begin{minipage}{12cm}
            \begin{spacing}{1.3}
                {\bullet} Solid training in mathematics, data sciences and artificial intelligence.\\
                {\bullet} Programming experience in Python, Java, Javascript, R, MATLAB, C ...\\
                %{\bullet} Proficiency in Python, TensorFlow, or PyTorch for development.\\
                {\bullet} Strong understanding of machine learning algorithms and architectures.\\
                {\bullet} Experience of deep learning models for CV, NLP, RL and LLMs.\\
                %{\bullet} Experience with generative models like GANs, VAEs, and transformers.\\
                %{\bullet} Advanced knowledge of probability theory and stochastic processes.\\
                %{\bullet} Ability to preprocess and manipulate large-scale datasets efficiently.\\
                {\bullet} Design of data acquisition, cleaning and transformation pipelines.\\
                %{\bullet} Familiarity with deep learning frameworks for model training and evaluation.\\
                {\bullet} Skill in optimizing models for performance and scalability.\\
                {\bullet} Experience in deploying AI models in production environments.\\
                %{\bullet} Expertise in evaluating and interpreting model results and performance metrics.\\
                {\bullet} Practice with relational, non-relational, and graph-oriented databases.\\
                {\bullet} Practice with orchestration, containerization and CI/CD.\\
                {\bullet} Practice with Big Data solutions: streaming, distributed processing.\\
                {\bullet} Practice with AWS cloud computing, sagemaker, terraform IaaC.\\     
                {\bullet} Strong problem-solving skills and adaptability to new technologies.\\           
                {\bullet} Technical and non-technical communication through various mediums.\\
                %: articles, conferences, reports, posters, websites, dashboards.\\
                {\bullet} Collaboration with multidisciplinary teams on various subjects.\\
                {\bullet} Research, technological and scientific watch.
                % {\bullet} Solid training in mathematics, data sciences and artificial intelligence.\\
                % {\bullet} Programming experience in Python, Java, Javascript, R, MATLAB, C ...\\
                % {\bullet} Design of data acquisition, cleaning, transformation, and storage pipelines.\\
                % {\bullet} Statistical modeling and machine learning for predictive model creation. \\
                % {\bullet} Implementation of deep learning techniques for computer vision, natural language processing, reinforcement learning.\\
                % {\bullet} Practive of Advanced techniques for generative AI.\\
                % %{\bullet} Practive of Advanced techniques for generative AI, finetuning, transfer learning, QLora, distillation...\\
                % %{\bullet} Maîtrise des langages de programmation tels que Python, R, SQL, Java, Scala, etc.\\
                % %{\bullet} Connaissance des outils de manipulation de données tels que Pandas, NumPy, Spark, etc.\\
                % %{\bullet} Capacité à concevoir et à mettre en œuvre des pipelines de données pour l'ingestion, le nettoyage, la transformation et le stockage de données.\\
                % {\bullet} Practice with relational, non-relational, and graph-oriented databases.\\
                % %{\bullet} Practice with Big Data solutions: streaming, distributed processing.\\
                % %{\bullet} Development of web applications for data visualization and user interaction.\\
                % %{\bullet} Development of APIs for data access and model deployment.\\
                % {\bullet} Pratice with AWS cloud computing solutions, sagemaker, terraform IaaS.\\                
                % %{\bullet} Compétences en informatique distribuée pour le traitement de données à grande échelle.\\
                % {\bullet} Practice with orchestration, containerization, continuous integration and deployment tools.\\
                % {\bullet} Technical and non-technical communication through various mediums: articles, conferences, reports, posters, websites, dashboards.\\
                % {\bullet} Collaboration with multidisciplinary teams on various subjects.\\
                % {\bullet} Research, technological and scientific watch.
            \end{spacing}
        \end{minipage}    
        %\vspace*{0.5cm}
        %{\bullet} Capacité à travailler avec des outils de visualisation de données\\

        % \section*{ COMPETENCES}
            
        %     \begin{multicols*}{2}
        %         \textbf{Data Science}\\
        %         \bullet \; Data analysis \& visualisation\\ 
        %         \bullet \; Data assimilation\\ 
        %         \bullet \; Machine Learning\\
        %         \bullet \; Deep learning\\
        %         \bullet \; Recommendation System\\
        %         \bullet \; Geophysical Fluid dynamics
        %         \vfill\null
        %         \columnbreak
        %         \textbf{Langages \& Outils}\\
        %         \bullet \; python, bash, \LaTeX, Git, Matlab\\
        %         \bullet \; pytorch, tensorflow, sklearn\\
        %         \bullet \; VS code, jupyter, Android Studio\\

        %         \textbf{Isolation \& Orchestration}\\
        %         \bullet \; Docker, Kubernetes, Airflow
        %     \end{multicols*}

        %     %\vspace*{0.22cm}
        %     \vspace*{-0.2cm}
        %     \begin{multicols*}{3}
        %         \textbf{Bases de données}\\
        %         \bullet \; SQL \& NoSQL\\ 
        %         \bullet \; MongoDB\\
        %         \bullet \; ElasticSearch\\
        %         \bullet \; Neo4j
        %         \vfill\null
        %         \columnbreak
        %         \textbf{Big Data}\\
        %         \bullet \; Java Spark\\
        %         \bullet \; Pyspark\\ 
        %         \bullet \; Hadoop Hive\\
        %         \bullet \; Kafka\\

        %         \textbf{Développement}\\
        %         \bullet \; Fastapi\\
        %         \bullet \; Flask\\
        %         \bullet \; Dash\\
        %         \bullet \; Streamlit\\
        %     \end{multicols*}
        
        % Work Experience
        \vspace*{-0.22cm}
        %\vspace*{0.22cm}
 
 
        \section*{EXPERIENCE}

        \textbf{Machine Learning Engineer} \hfill Aug 2023\--present\\
        \textbf{4.71} \-- full-remote \hfill Clermont Ferrand, FR\\

        \vspace*{-0.2cm}
        \hspace*{0.2cm}
        \begin{minipage}{12cm}
            {\bullet} Developing artificial intelligence techniques for calculating companies' carbon footprint based on their general accounting.\\
            {\bullet} Application of deep learning models for document analysis.\\
            {\bullet} Grant proposal writing for research projects.\\
        \end{minipage}

        \textbf{Visiting Research Fellow} \hfill 2020\-- Sep 2023\\
        \textbf{University of Surrey} \-- full-remote \hfill Guildford, UK\\

        \vspace*{-0.2cm}
        \hspace*{0.2cm}
        \begin{minipage}{12cm}
            {\bullet} \textbf{data Science/Engineering}: \href{https://sdelahaies.github.io/enpgf-lab.html}{Online training for temporal point processes using pytorch and kafka.}\\
        \end{minipage}

        \textbf{Post-doctoral Research Assistant} \hfill 2009\--2020\\
        \textbf{\href{https://www.surrey.ac.uk/department-mathematics}{University of Surrey}} \-- full-remote\hfill Guildford, UK\\
        $\star$ \textbf{Data Assimilation and Stochastic Point Processes}\\
        
        \vspace*{-0.2cm}
        \hspace*{0.2cm}
        \begin{minipage}{12cm}
            {\bullet} Development of Bayesian filters to analyse and model crime, social and neural networks, Covid19 outbreak...\\
            {\bullet} Insurance risks associated with land subsidence.\\
            {\bullet} Development of a web app to monitor the evolution of the Covid19 outbreak in UK.\\
        \end{minipage}

        $\star$ \textbf{Data Assimilation and Terrestrial Ecosystems}\\
        
        \vspace*{-0.2cm}
        \hspace*{0.2cm}
        \begin{minipage}{12cm}
            {\bullet} Development of variational methods for the analysis of the Carbon cycle for terrestrial ecosystems.\\
            {\bullet} Study of very high resolution satellite imagery together with a terrestrial ecosystem Carbon cycle model to monitor and detect forest degradation.\\
            {\bullet} Development of variational tools to evaluate the information content of different observation fluxes in Carbon cycle models. \\
        \end{minipage}

        $\star$ \textbf{Teaching:} Numerical soultions for PDEs \hfill 2011\--2012\\
        %$\star$ \textbf{Communication:} \href{https://www.researchgate.net/profile/Sylvain-Delahaies-2/}{publications, posters, conferences, seminars.}        
    \end{minipage}} %

\end{document}
